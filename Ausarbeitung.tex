\documentclass[12pt,a4paper, listof=entryprefix, bibliography=totocnumbered,toc=listofnumbered,footsepline]{scrartcl}

% Fonts / Encoding
\usepackage{microtype}
\usepackage{lmodern}
\usepackage[T1]{fontenc}
\usepackage[utf8]{inputenc}
\usepackage[ngerman]{babel}
\usepackage{upquote}
\usepackage{csquotes}

% Mathe
\usepackage[fleqn]{amsmath}
\usepackage{amssymb}
\usepackage{amsfonts}
\usepackage{amsthm}
\usepackage{latexsym}
\usepackage{nccmath}

% Grafik
\usepackage[usenames,dvipsnames]{color}
\usepackage{colortbl}
\usepackage{graphicx}
\usepackage{subfig}
\usepackage[table]{xcolor}

% Quellcode
\usepackage{listings}
\usepackage{scrhack}

% Seitengestaltung
\usepackage{array}
\usepackage{float}
\usepackage{geometry}
\usepackage{setspace}
\usepackage[automark,headsepline]{scrlayer-scrpage}

% Tabellen
\usepackage{hhline}
\usepackage{multirow}

% Aufzählungen
\usepackage{paralist}

% PDF
\usepackage{hyperref}
\usepackage{url}

% Verzeichnisse
\usepackage[printonlyused]{acronym}
\usepackage[nonumberlist, nogroupskip]{glossaries}

% LEGACY
%\usepackage{fancyhdr}
\usepackage{titlesec}

%-----------------------------------------------------------------------------------
% Bibilothek
%-----------------------------------------------------------------------------------
% Einbinden des BibLateX paketes mit Ausgabeeinstellungen
\usepackage[
style=alphabetic,          % Zitierstil
maxbibnames=50,            % alle Autorennamen anzeigen
maxcitenames=4,            % maximale Namen, die im Kürzel angezeigt werden
autocite=inline,           % regelt Aussehen für \autocite (inline=\parancite)
block=space,               % kleiner horizontaler Platz zwischen den Feldern
backref=true,              % Seiten anzeigen, auf denen die Referenz vorkommt
backrefstyle=three+,       % fasst Seiten zusammen, z.B. S. 2f, 6ff, 7-10
date=short,                % Datumsformat
backend = biber,           % Backnend für Aufbereitung
]{biblatex}

%Zusätzliche für Umbrüche für Kleinbuchstaben z.B. in URLs
\appto\UrlBreaks{\do\a\do\b\do\c\do\d\do\e\do\f\do\g\do\h\do\i\do\j
	\do\k\do\l\do\m\do\n\do\o\do\p\do\q\do\r\do\s\do\t\do\u\do\v\do\w
	\do\x\do\y\do\z}


\newcounter{verzeichnis}
\setcounter{verzeichnis}{1}

%Abstände der Einträge
\setlength{\bibitemsep}{1em}     % Abstand zwischen den Literaturangaben
\setlength{\bibhang}{2em}        % Einzug nach jeweils erster Zeile


% Kürzel soll vier Buchstaben der Autoren enthalten statt drei
\DeclareLabelalphaTemplate{
	\labelelement{
		\field[final]{shorthand}
		\field{label}
		\field[strwidth=4,strside=left,ifnames=1]{labelname}
		\field[strwidth=2,strside=left,ifnames=2]{labelname}
		\field[strwidth=1,strside=left]{labelname}
	}
	\labelelement{
		\field[strwidth=2,strside=right]{year}
	}
}

% Bibliothek der Quellen
\addbibresource{bib.bib}

% --------------------------------------------------------------------------------
% Einstellung für Listings
% --------------------------------------------------------------------------------
\lstset{
	basicstyle=\footnotesize, 
	captionpos=b,
	breaklines=true,
	showstringspaces=false,
	tabsize=2,
	frame=lines,
	numbers=left,
	numberstyle=\tiny,
	xleftmargin=2em,
	framexleftmargin=2em
}
\makeatletter
\def\l@lstlisting#1#2{\@dottedtocline{1}{0em}{1em}{\hspace{1,5em} Lst. #1}{#2}}
\makeatother

% --------------------------------------------------------------------------------
% Seitenformate
% --------------------------------------------------------------------------------
%Seitenformat
\geometry{a4paper, top=27mm, left=30mm, right=20mm, bottom=32mm, headsep=12mm, footskip=10mm}

% --------------------------------------------------------------------------------
% Metainformationen
% --------------------------------------------------------------------------------
\hypersetup{
	pdftitle={Ausarbeitung},
	pdfauthor={Tobias Mehrl, Thomas Butz},
	pdfsubject={Unterschied zwischen Normal- und Gammaverteilung},
	pdfcreator={\LaTeX\ with package \flqq hyperref\frqq},
	pdfproducer={pdfTeX \the\pdftexversion.\pdftexrevision},
	pdfnewwindow=true,
	colorlinks=true,
	linkcolor=black,
	citecolor=black,
	filecolor=magenta,
	urlcolor=black
}

%-----------------------------------------------------------------------------------
% Abkürzungen AKRONYME HIER ERGÄNZEN
%-----------------------------------------------------------------------------------
\glssetwidest{OTHR}% Längste Abkürzung für eine korrekte Einrückung

\makenoidxglossaries %Leeres Verzeichnis erstellen

%Abkürzungen hinzufügen
\newacronym{LIP}{LIP}{Labor für Informationstechnik und Produktionslogistik}
\newacronym{OTH}{OTHR}{Ostbayerische Technische Hochschule Regensburg}
\newacronym{MRP}{MRP}{Material Requirements Planing}

\begin{document}
% --------------------------------------------------------------------------------
% Globale Formateinstellungen
% --------------------------------------------------------------------------------
\onehalfspacing
% Abstände Überschrift
\titlespacing{\section}{0pt}{42pt}{6pt}
\titlespacing{\subsection}{0pt}{12pt}{6pt}
\titlespacing{\subsubsection}{0pt}{12pt}{6pt}

% Kopf- und Fusszeile
\clearpairofpagestyles
\ofoot[\pagemark]{\pagemark}
\lehead{\headmark}
\rohead{\headmark}
\pagestyle{scrheadings}

% Nummereriung
\renewcommand{\thesection}{\Roman{section}}
\renewcommand{\theHsection}{\Roman{section}}
\pagenumbering{Roman}

% eigene Farbdefinitionen
\definecolor{lip}{HTML}{3366FF}
\definecolor{grey}{HTML}{ABABAB}

% ---------------------------------------------------------------------------
% Titelseite
% ---------------------------------------------------------------------------
\thispagestyle{empty}

%LIP Schriftzug in eigener Farbe 
\textsf{\begin{minipage}{.60\textwidth}
	\large
	\textcolor{lip}{\textbf{Labor für Informationstechnik und\\Produktionslogistik (LIP)}} %Farbe setzten
	\small 
	\textbf{\\Fortgeschrittene Produktionsplanung}
	\\Professor Dr.-Ing. Frank Herrmann
\end{minipage}
%Einbinden des OTH Logos mir rechtsbündiger Ausrichtung
\begin{minipage}{.29\textwidth}
	\begin{flushright}
		\includegraphics[scale=.15]{Bilder/HS-Logo.jpg}\\
	\end{flushright}
\end{minipage}}
 
% Zeilenabstand
\onehalfspacing	

%Beschriftung der Titelseite
\begin{center}

	\vspace*{4cm} %4 cm Vorspann
	\Large
	\textbf{Vergleich von Normal- und Gammaverteilung}\\ %Titel der Arbeit
	\large
	\textbf{im Bestandsmanagement}\\ %Untertitel der Arbeit
		
	\vspace*{8cm} %8 cm Vorspann
	\normalsize
	\begin{center}
	\date{\today}
	\textbf{Tobias Mehrl (B.Sc)}\\ %Name des Autors
	\textbf{Thomas Butz  (B.Sc)}
	\end{center}
\end{center}
\pagebreak

% ------------------------------------------------------------------------------
% Inhaltsverzeichnis
% ------------------------------------------------------------------------------
% Inhaltsverzeichnis
\singlespacing %Zeilenabsatnd reduzieren
\setcounter{section}{0}
\setcounter{page}{1}
\addcontentsline{toc}{section}{Inhaltsverzeichnis}%hinzufügen des Inhaltsverzeichnises selbst

\tableofcontents %Ausgabe des Inhaltsverzeichnisses
\pagebreak

% ------------------------------------------------------------------------------
% Setzen der Nummerierungen für Normaltext
% ------------------------------------------------------------------------------
\onehalfspacing %Zeilenabstand auf 1.5
\renewcommand{\thesection}{\arabic{section}} %Arabische Beschriftung für Absatznummern
\pagenumbering{arabic}  %Seitennummerrierung auf arabisch setzten
\setcounter{page}{1}	%Seitenzahl für Inhalt auf 1 setzten
\setcounter{section}{0}
% Kopfzeile mit aktuellem Hauptkapitel darstellen
\renewcommand{\sectionmark}[1]{\markright{#1}} %Section ausgeben
\renewcommand{\subsectionmark}[1]{}            %Subsection nicht ausgeben
\renewcommand{\subsubsectionmark}[1]{}         %Subsubsection nicht ausgeben
%\rhead{\rightmark}                             %Ausgabe Rechtsbündig

%------------------------------------------------------------------------------
%	Inhalt
%------------------------------------------------------------------------------

\section{Aufgabenstellung}
Das Unternehmen Conrad verkauft seit kurzem ein neuartiges Smartphone Modell. Der Absatz an diesen Geräten wird über eine Verteilungsfunktion mit einem Erwartungswert von 60 und einer zugehörigen Varianz von 700 Stück angegeben. Des Weiteren soll ein $\beta$-Servicegrad von 0,95 umgesetzt werden. Die Bestellmenge \(q\) sei fest mit 100 Stück vorgegeben.

Berechnen Sie die optimalen Bestellpunkte $s_{opt}$ je unter der Annahme, dass es sich bei der Verteilungsfunktion in einem Fall um eine Normal- und im anderen um eine Gammaverteilung handelt.
\section{Formeln}
Der folgende Punkt nennt die zur Berechnung benötigten Variablen und Formeln.
\subsection{Definitionen}
\(Y^*\) = Zufallsvariable für die Nachfragemenge

\(E(Y^*)\) = Erwartungswert der Zufallsvariable $Y$

\(Var(Y^*) \equiv \sigma(Y^*)\) = Varianz der Zufallsvariable $Y$

\(s_{opt}\) = optimaler Bestellpunkt

\(q_{opt}\) = optimale Bestellmenge

\(I^{f, t}_{x}(s)\) = Fehlbestand zum Zeitpunkt \(t\) für den Bestellpunkt \(s\) und einer Verteilungsfunktion \(x\) 

\(F_x(s)\) = Fehlbestand für den Bestellpunkt \(s\) mit Verteilungsfunktion x 

\(\mathcal{N}_{(0,1)}\) = Standardnormalverteilung

\(\Gamma(\alpha_{Y^*}, k_{Y^*})\) = Gammaverteilung mit Skalenparameter \(\alpha\) und Formparameter \(k\)

\(\Phi_x\) = Dichtefunktion für die Verteilungsfunktion x

\subsection{Allgemein}
\begin{gather*}
\beta = 1-\frac{E\left(F_{x}(s)\right)}{q_{opt}} \text{ für eine Verteilungsfunktion x}
\end{gather*}
\subsection{Normalverteilung}
\begin{gather*}
E\left(F_{\mathcal{N}(0,1)}(s)\right)=\Phi^1_{\mathcal{N}(0,1)}\left(\frac{s-E(Y^*)}{\sigma(Y^*)}\right)-\Phi^1_{\mathcal{N}(0,1)}\left(\frac{s+q_{opt}-E(Y^*)}{\sigma(Y^*)}\right) \\
%s_{opt} = E(Y^*)+v_{s_{opt}} \cdot \sigma(Y^*) \Leftrightarrow v_{s_{opt}}=\frac{s_{opt}-E(Y^*)}{\sigma(Y^*)} \\
%\beta = 1-\frac{\Phi^1_{\mathcal{N}(E(Y^*),\sigma^2(Y^*))}(s_{opt})-\Phi^1_{\mathcal{N}(E(Y^*),\sigma^2(Y^*))}(s_{opt}+q_{opt})}{q_{opt}}
\end{gather*}
\subsection{Gammaverteilung}
\begin{gather*}
E\left(F_{\Gamma\left(\alpha_{Y^*},k_{Y^*}\right)}(s)\right)= E\left(I^{f,End}_{\Gamma\left(\alpha_{Y^*},k_{Y^*}\right)}\left(s\right)\right)-E\left(I^{f,Anf}_{\Gamma\left(\alpha_{Y^*},k_{Y^*}\right)}\left(s\right)\right) \\
E\left(I^{f,End}_{\Gamma\left(\alpha_{Y^*},k_{Y^*}\right)}\left(s_{opt}\right)\right)=\frac{k_{Y^*}}{\alpha_{Y^*}}-s_{opt}-\frac{k_{Y^*}}{\alpha_{Y^*}}\cdot \Phi_{\Gamma\left(\alpha_{Y^*},k_{Y^*}+1\right)}\left(s_{opt}\right)+s_{opt}\cdot \Phi_{\Gamma\left(\alpha_{Y^*},k_{Y^*}\right)}\left(s_{opt}\right) \\
E\left(I^{f,Anf}_{\Gamma\left(\alpha_{Y^*},k_{Y^*}\right)}\left(s_{opt}\right)\right)=\frac{k_{Y^*}}{\alpha_{Y^*}}-\left(s_{opt}+q_{opt}\right)-\frac{k_{Y^*}}{\alpha_{Y^*}}\cdot \Phi_{\Gamma\left(\alpha_{Y^*},k_{Y^*}+1\right)}\left(s_{opt}+q_{opt}\right) \\
\left(1-\beta\right)\cdot q_{opt}= E\left(F_{\Gamma\left(\alpha_{Y^*},k_{Y^*}\right)}\left(s_{opt}\right)\right)
\end{gather*}
\section{Lösung}

Gesucht sind die beiden Werte für \(s_{opt}\), die sich bei Einhaltung des \(\beta\)-Servicegrades für die beiden Verteilungsfunktionen \(\Gamma(\alpha_{Y^*},k_{Y^*})\) und \(\mathcal{N}(0,1)\) ergeben.

Hierfür sind folgende von der Aufgabenstellung vorgegebenen Werte zu betrachten:
\begin{itemize}
	\item \(E(Y^*)=60\text{ Stück}\)
	\item \(Var(Y^*)\equiv\sigma(Y^*)=700\)
	\item \(\beta\text{-Servicegrad}=0,95\)
	\item \(q=100\text{ Stück}\) 
\end{itemize}

Der Wert für \(s_{opt}\) lässt sich durch eine Optimierungsstrategie bestimmen. Dabei entspricht \(s_{opt}\) dem niedrigsten Wert von \(s\) bei dem der vorgegebene Servicegrad \(\beta\) noch erreicht wird. 

Die dafür maximal zulässige  zu erwartende Fehlmenge bei einer zugrunde liegenden Normalverteilung, lässt sich über die zuvor genannte Formel für den \(\beta\)-Servicegrad berechnen:

\[\beta = 1-\frac{E\left(F_{\mathcal{N}(0,1)}(s)\right)}{q_{opt}} \Leftrightarrow E\left(F_{\mathcal{N}(0,1)}(s)\right)=(1-\beta)\cdot q_{opt}\]

Durch Einsetzen der vorgegebenen Werte für \(q\) und \(\beta\)-Servicegrad in die obige Gleichung, ergibt sich eine maximal zulässige Fehlmenge von \((1-0,95)\cdot 100 \text{ Stück}=5\text{ Stück}\). Dieser Wert gilt ebenfalls bei einer gammaverteilten Zufallsvariable.

Um den optimalen Bestellpunkt \(s_{opt}\) ermitteln zu können, müssen daher für verschiedene Werte von \(s\) die sich dadurch ergebenden Fehlmengen betrachtet werden und möglichst nahe an den zuvor berechneten Wert von 5 Stück angenähert werden.

Für die Normalverteilung kann der Erwartungswert für die Fehlmenge direkt durch die nachfolgende Gleichung berechnet werden. 

\[E\left(F_{\mathcal{N}(0,1)}(s)\right)=\Phi^1_{\mathcal{N}(0,1)}\left(\frac{s-E(Y^*)}{\sigma(Y^*)}\right)-\Phi^1_{\mathcal{N}(0,1)}\left(\frac{s+q_{opt}-E(Y^*)}{\sigma(Y^*)}\right)\]

Beim Einsetzen eines exemplarischen Wertes von 75 Stück für s, ergibt sich:

\begin{align*}
E\left(F_{\mathcal{N}(0,1)}(75)\right)&=\Phi^1_{\mathcal{N}(0,1)}\left(\frac{(75-60)\text{ Stk.}}{700}\right)-\Phi^1_{\mathcal{N}(0,1)}\left(\frac{(75+100-60)\text{ Stk.}}{700}\right)\\
&\approx4,7\text{ Stk.}
\end{align*}

\pagebreak
Analog dazu lässt sich der Erwartungswert für die Fehlmenge bei einer gammaverteilten Zufallsvariable wie folgt berechnen:

\begin{align*}
&E\left(F_{\Gamma\left(\alpha_{Y^*},k_{Y^*}\right)}(s)\right)= E\left(I^{f,End}_{\Gamma\left(\alpha_{Y^*},k_{Y^*}\right)}\left(s\right)\right)-E\left(I^{f,Anf}_{\Gamma\left(\alpha_{Y^*},k_{Y^*}\right)}\left(s\right)\right) = \\
&= E\left(I^{f,End}_{\Gamma\left(\alpha_{Y^*},k_{Y^*}\right)}\left(s\right)\right)-E\left(I^{f,Anf}_{\Gamma\left(\alpha_{Y^*},k_{Y^*}\right)}\left(s\right)\right)= \\
&= \frac{k_{Y^*}}{\alpha_{Y^*}}-s-\frac{k_{Y^*}}{\alpha_{Y^*}}\cdot \Phi_{\Gamma\left(\alpha_{Y^*},k_{Y^*}+1\right)}\left(s\right)+s\cdot \Phi_{\Gamma\left(\alpha_{Y^*},k_{Y^*}\right)}\left(s\right) \\
&\quad - \left(\frac{k_{Y^*}}{\alpha_{Y^*}}-\left(s+q_{opt}\right)-\frac{k_{Y^*}}{\alpha_{Y^*}}\cdot \Phi_{\Gamma\left(\alpha_{Y^*},k_{Y^*}+1\right)}\left(s+q_{opt}\right)\right) = \\
&= q_{opt}+s \cdot \Phi_{\Gamma\left(\alpha_{Y^*},k_{Y^*}+1\right)}\left(s\right)- \frac{k_{Y^*}}{\alpha_{Y^*}} \cdot \Phi_{\Gamma\left(\alpha_{Y^*},k_{Y^*}\right)}\left(s\right) + \frac{k_{Y^*}}{\alpha_{Y^*}} \cdot \Phi_{\Gamma\left(\alpha_{Y^*},k_{Y^*}+1\right)}\left(s+q_{opt}\right)
\end{align*}


Exemplarisch für einen Wert s von 75 Stück wird nun die Fehlmenge für eine Gammaverteilung wie folgt berechnet:

\begin{align*}
&E\left(F_{\Gamma\left(\alpha_{Y^*},k_{Y^*}\right)}(75 \text{ Stk.})\right)= \\
&= 100\text{ Stk.}+75\text{ Stk.} \cdot \Phi_{\Gamma\left(\alpha_{Y^*},k_{Y^*}+1\right)}\left(75\text{ Stk.}\right)- \frac{k_{Y^*}}{\alpha_{Y^*}} \cdot \Phi_{\Gamma\left(\alpha_{Y^*},k_{Y^*}\right)}\left(75\text{ Stk.}\right) \\
&\quad + \frac{k_{Y^*}}{\alpha_{Y^*}} \cdot \Phi_{\Gamma\left(\alpha_{Y^*},k_{Y^*}+1\right)}\left(75\text{ Stk.}+100\text{ Stk.}\right)=
\end{align*}
\subsection{Auswertung}
\begin{table}[!h]
	\centering
	\begin{tabular}{|c||c|c|c||c|c|c|}
		\hline
		\multirow{2}{*}{s} & \multicolumn{3}{c||}{Gammaverteilung} & \multicolumn{3}{c|}{Normalverteilung}\\
		\cline{2-7}
		 & \(E\left(F_{end}\right)\) & \(E\left(F_{anf}\right)\) & \(E\left(F\right)\) & \(E\left(F_{end}\right)\) & \(E\left(F_{anf}\right)\) & \(E\left(F\right)\)  \\
		\hline
		73,500 & 5,6650 & 0,0173 & 5,648 & 5,1500 & 0 & 5,150 \\
		73,600 & 5,6384 & 0,0172 & 5,621 & 5,1196 & 0 & 5,120 \\
		73,700 & 5,6118 & 0,0171 & 5,595 & 5,0893 & 0 & 5,089 \\
		73,800 & 5,5854 & 0,0170 & 5,568 & 5,0591 & 0 & 5,059 \\
		73,900 & 5,5590 & 0,0169 & 5,542 & 5,0291 & 0 & 5,029 \\
		73,950 & 5,5459 & 0,0168 & 5,529 & 5,0141 & 0 & 5,014 \\
		73,980 & 5,5380 & 0,0168 & 5,521 & 5,0052 & 0 & 5,005 \\
		\hline
		73,996 & 5,5339 & 0,0168 & 5,517 & 5,0004 & 0 & \cellcolor[gray]{0.8} 5,000 \\
		\hline
		74,020 & 5,5276 & 0,0168 & 5,511 & 4,9932 & 0 & 4,993 \\
		74,050 & 5,5197 & 0,0167 & 5,503 & 4,9843 & 0 & 4,984 \\
		74,080 & 5,5119 & 0,0167 & 5,495 & 4,9754 & 0 & 4,975 \\
		74,100 & 5,5067 & 0,0167 & 5,490 & 4,9694 & 0 & 4,969 \\
		74,200 & 5,4807 & 0,0166 & 5,464 & 4,9398 & 0 & 4,940 \\
		74,300 & 5,4547 & 0,0165 & 5,438 & 4,9103 & 0 & 4,910 \\
		74,400 & 5,4289 & 0,0163 & 5,413 & 4,8809 & 0 & 4,881 \\
		74,500 & 5,4032 & 0,0162 & 5,387 & 4,8516 & 0 & 4,852 \\
		74,600 & 5,3776 & 0,0161 & 5,362 & 4,8225 & 0 & 4,823 \\
		74,700 & 5,3521 & 0,0160 & 5,336 & 4,7935 & 0 & 4,794 \\
		74,800 & 5,3268 & 0,0159 & 5,311 & 4,7647 & 0 & 4,765 \\
		74,900 & 5,3015 & 0,0158 & 5,286 & 4,7359 & 0 & 4,736 \\
		75,000 & 5,2763 & 0,0157 & 5,261 & 4,7073 & 0 & 4,707 \\
		75,100 & 5,2512 & 0,0156 & 5,236 & 4,6789 & 0 & 4,679 \\
		75,200 & 5,2262 & 0,0155 & 5,211 & 4,6505 & 0 & 4,651 \\
		75,300 & 5,2014 & 0,0154 & 5,186 & 4,6223 & 0 & 4,622 \\
		75,400 & 5,1766 & 0,0153 & 5,161 & 4,5942 & 0 & 4,594 \\
		75,500 & 5,1519 & 0,0152 & 5,137 & 4,5663 & 0 & 4,566 \\
		75,600 & 5,1274 & 0,0151 & 5,112 & 4,5384 & 0 & 4,538 \\
		75,700 & 5,1029 & 0,0150 & 5,088 & 4,5107 & 0 & 4,511 \\
		75,800 & 5,0785 & 0,0149 & 5,064 & 4,4831 & 0 & 4,483 \\
		75,900 & 5,0543 & 0,0148 & 5,039 & 4,4557 & 0 & 4,456 \\
		76,020 & 5,0253 & 0,0147 & 5,011 & 4,4229 & 0 & 4,423 \\
		\hline
		76,066 & 5,0142 & 0,0147 & \cellcolor[gray]{0.8} 5,000 & 4,4104 & 0 & 4,410 \\
		\hline
		76,086 & 5,0094 & 0,0146 & 4,995 & 4,4049 & 0 & 4,405 \\
		76,100 & 5,0060 & 0,0146 & 4,991 & 4,4011 & 0 & 4,401 \\
		76,300 & 4,9582 & 0,0144 & 4,944 & 4,3471 & 0 & 4,347 \\
		76,400 & 4,9345 & 0,0143 & 4,920 & 4,3203 & 0 & 4,320 \\
		76,500 & 4,9108 & 0,0142 & 4,897 & 4,2936 & 0 & 4,294 \\
		\hline
	\end{tabular}
	\caption{Ermittlung von $s_{opt}$ über den Erwartungswert der Fehlmenge}
	\label{tab:auswertung}
\end{table}
\pagebreak

Stellt man die Tabelle \ref{tab:auswertung} mit äquidistanten Werten für s in Form eines Diagramms dar, so ergibt sich die Grafik \ref{img:vergleich}. Die Markierungen kennzeichnen die gesuchten Zielwerte für die beiden Verteilungsfunktionen. Die zugehörigen x-Werte der Markierungen entsprechen den Werten für \(s_{opt}\).

Es kann beobachtet werden, dass die Normalverteilung im Vergleich zur Gammaverteilung einen um \(76,1\text{ Stk.}-74,0\text{ Stk.}=2,1\text{ Stk.}\) geringeren optimalen Bestellpunkt besitzt.

\begin{figure}
	\centering
	\includegraphics[width=\textwidth,trim=4cm 1.5cm 4cm 1.5cm, clip=true]{./Bilder/VerteilungsfunktionenVerg.pdf}
	\caption{Vergleich der zu erw. Fehlmengen in Abhängigkeit von s}
	\label{img:vergleich}
\end{figure}

\pagebreak

%-------------------------------------------------------------------------------------
% Verzeichnisse %-------------------------------------------------------------------------------------
	\markboth{Verzeichnisse}{Verzeichnisse} %Kopftextbeschriftung

	\stepcounter{section}
	\phantomsection \label{Verzeichnisse}
	\addcontentsline{toc}{section}{Verzeichnisse} %Ohne Nummer ins Inhaltsverzeichnis
	\renewcommand{\thesection}{\Roman{verzeichnis}}
	\section*{Verzeichnisse}
	% Literaturverzeichnis
	\markboth{Verzeichnisse}{Verzeichnisse}
	\renewcommand{\refname}{Literaturverzeichnis}
	\printbibliography
	\pagebreak
	
	% Abbildungsverzeichnis
	\stepcounter{verzeichnis}
	\listoffigures
	\pagebreak
	
	% Tabellenverzeichnis
	\stepcounter{verzeichnis}
	\listoftables
	\pagebreak
	
	% Abkürzungen
%	\stepcounter{verzeichnis}
%	\section{Abkürzungsverzeichnis}
%	\vspace{-6em} % Abstand analog der anderen Verzeichnisse reduzieren
%	\printnoidxglossary[type=\acronymtype,style=alttree,title=,toctitle=] %automatischen Titel und Gliederungsbeschriftung unterdrücken - sonst steht da Glossarie
%	\newpage

% ---------------------------------------------------------------------------------
% Anhang
% ---------------------------------------------------------------------------------
%\pagenumbering{Roman} %Seitennummerierung auf Romanisch stellen
%\setcounter{page}{1}  %Seitenzähler initialisieren
%\markboth{Anhang}{Anhang}        %Kopftextbeschriftung
%\begin{appendix}
%	\phantomsection \label{Anhang}
%	\section*{Anhang}\addcontentsline{toc}{section}{Anhang} %Anhang ohne Nummer
%	\renewcommand{\thesection}{\Roman{section}}
%	
%	\setcounter{section}{0} %Counter für Nummerirung der Anhänge initalisieren
%	\section{GUI}
%	\label{app:gui}
%	Ein toller Anhang.
%	
%	\section{Screenshot}
%	\label{app:screenshot}
%	Zweiter Anhang.
%\end{appendix}

\end{document}
