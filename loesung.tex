\section{Lösung}

Gesucht sind die beiden Werte für \(s_{opt}\), die sich bei Einhaltung des \(\beta\)-Servicegrades für die beiden Verteilungsfunktionen \(\Gamma(\alpha_{Y^*},k_{Y^*})\) und \(\mathcal{N}(0,1)\) ergeben.

Hierfür sind folgende von der Aufgabenstellung vorgegebenen Werte zu betrachten:
\begin{itemize}
	\item \(E(Y^*)=60\text{ Stück}\)
	\item \(Var(Y^*)\equiv\sigma(Y^*)=700\)
	\item \(\beta\text{-Servicegrad}=0,95\)
	\item \(q=100\text{ Stück}\) 
\end{itemize}

Der Wert für \(s_{opt}\) lässt sich durch eine Optimierungsstrategie bestimmen. Dabei entspricht \(s_{opt}\) dem niedrigsten Wert von \(s\) bei dem der vorgegebene Servicegrad \(\beta\) noch erreicht wird. 

Die dafür maximal zulässige  zu erwartende Fehlmenge bei einer zugrunde liegenden Normalverteilung, lässt sich über die zuvor genannte Formel für den \(\beta\)-Servicegrad berechnen:

\[\beta = 1-\frac{E\left(F_{\mathcal{N}(0,1)}(s)\right)}{q_{opt}} \Leftrightarrow E\left(F_{\mathcal{N}(0,1)}(s)\right)=(1-\beta)\cdot q_{opt}\]

Durch Einsetzen der vorgegebenen Werte für \(q\) und \(\beta\)-Servicegrad in die obige Gleichung, ergibt sich eine maximal zulässige Fehlmenge von \((1-0,95)\cdot 100 \text{ Stück}=5\text{ Stück}\). Dieser Wert gilt ebenfalls bei einer gammaverteilten Zufallsvariable.

Um den optimalen Bestellpunkt \(s_{opt}\) ermitteln zu können, müssen daher für verschiedene Werte von \(s\) die sich dadurch ergebenden Fehlmengen betrachtet werden und möglichst nahe an den zuvor berechneten Wert von 5 Stück angenähert werden.

\subsection{Berechnung der Fehlmengen-Erwartungswerte}
Für die Normalverteilung kann der Erwartungswert für die Fehlmenge direkt durch die nachfolgende Gleichung berechnet werden. 

\[E\left(F_{\mathcal{N}(0,1)}(s)\right)=\Phi^1_{\mathcal{N}(0,1)}\left(\frac{s-E(Y^*)}{\sigma(Y^*)}\right)-\Phi^1_{\mathcal{N}(0,1)}\left(\frac{s+q_{opt}-E(Y^*)}{\sigma(Y^*)}\right)\]

Beim Einsetzen eines exemplarischen Wertes von 75 Stück für s, ergibt sich:

\begin{align*}
E\left(F_{\mathcal{N}(0,1)}(75)\right)&=\Phi^1_{\mathcal{N}(0,1)}\left(\frac{(75-60)\text{ Stk.}}{700}\right)-\Phi^1_{\mathcal{N}(0,1)}\left(\frac{(75+100-60)\text{ Stk.}}{700}\right)\\
&\approx4,7\text{ Stk.}
\end{align*}

\pagebreak
Analog dazu lässt sich der Erwartungswert für die Fehlmenge bei einer gammaverteilten Zufallsvariable wie folgt berechnen:

\begin{align*}
&E\left(F_{\Gamma\left(\alpha_{Y^*},k_{Y^*}\right)}(s)\right)= E\left(I^{f,End}_{\Gamma\left(\alpha_{Y^*},k_{Y^*}\right)}\left(s\right)\right)-E\left(I^{f,Anf}_{\Gamma\left(\alpha_{Y^*},k_{Y^*}\right)}\left(s\right)\right) = \\
&= E\left(I^{f,End}_{\Gamma\left(\alpha_{Y^*},k_{Y^*}\right)}\left(s\right)\right)-E\left(I^{f,Anf}_{\Gamma\left(\alpha_{Y^*},k_{Y^*}\right)}\left(s\right)\right)= \\
&= \frac{k_{Y^*}}{\alpha_{Y^*}}-s-\frac{k_{Y^*}}{\alpha_{Y^*}}\cdot \Phi_{\Gamma\left(\alpha_{Y^*},k_{Y^*}+1\right)}\left(s\right)+s\cdot \Phi_{\Gamma\left(\alpha_{Y^*},k_{Y^*}\right)}\left(s\right) \\
&\quad - \left(\frac{k_{Y^*}}{\alpha_{Y^*}}-\left(s+q_{opt}\right)-\frac{k_{Y^*}}{\alpha_{Y^*}}\cdot \Phi_{\Gamma\left(\alpha_{Y^*},k_{Y^*}+1\right)}\left(s+q_{opt}\right)\right) = \\
&= q_{opt}+s \cdot \Phi_{\Gamma\left(\alpha_{Y^*},k_{Y^*}+1\right)}\left(s\right)- \frac{k_{Y^*}}{\alpha_{Y^*}} \cdot \Phi_{\Gamma\left(\alpha_{Y^*},k_{Y^*}\right)}\left(s\right) + \frac{k_{Y^*}}{\alpha_{Y^*}} \cdot \Phi_{\Gamma\left(\alpha_{Y^*},k_{Y^*}+1\right)}\left(s+q_{opt}\right)=\\
&= q_{opt}+s \cdot \Phi_{\Gamma\left(\alpha_{Y^*},k_{Y^*}+1\right)}\left(s\right)- E(Y^*) \cdot \Phi_{\Gamma\left(\alpha_{Y^*},k_{Y^*}\right)}\left(s\right) + E(Y^*) \cdot \Phi_{\Gamma\left(\alpha_{Y^*},k_{Y^*}+1\right)}\left(s+q_{opt}\right)
\end{align*}


Exemplarisch für einen Wert s von 75 Stück wird nun die Fehlmenge für eine Gammaverteilung wie folgt berechnet:

\begin{align*}
&E\left(F_{\Gamma\left(\alpha_{Y^*},k_{Y^*}\right)}(75 \text{ Stk.})\right)= \\
&= 100\text{ Stk.}+75\text{ Stk.} \cdot \Phi_{\Gamma\left(\alpha_{Y^*},k_{Y^*}+1\right)}\left(75\text{ Stk.}\right)- 60\text{Stk.} \cdot \Phi_{\Gamma\left(\alpha_{Y^*},k_{Y^*}\right)}\left(75\text{ Stk.}\right) \\
&\quad + 60\text{ Stk.} \cdot \Phi_{\Gamma\left(\alpha_{Y^*},k_{Y^*}+1\right)}\left(75\text{ Stk.}+100\text{ Stk.}\right) \approx 5,3\text{ Stk.}
\end{align*}

Basierend auf den exemplarisch berechneten Werten kann nun \(s_{opt}\) über Annäherung der Fehlmenge an den oben berechneten Zielwert von 5,0 Stk. ermittelt werden. Zu beachten ist dabei, dass die Funktionen zur Berechnung der Erwartungswerte für die beiden Verteilungsfunktionen der Fehlmengen in beiden Fällen streng monoton fallend sind.

Für die Normalverteilung muss der Wert von s daher stückweise verringert werden, bis die gesuchte Fehlmenge erreicht ist. Im Gegensatz dazu muss s im Falle der Gammaverteilung erhöht werden, um \(s_{opt}\) zu erhalten. Die Tabelle \ref{tab:auswertung} führt die Ergebnisse dieses Vorgehens parallel für beide Verteilungsfunktionen auf und zeigt die so gefundenen Werte von \(s_{opt}\).