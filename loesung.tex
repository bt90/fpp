\section{Lösung}

Gesucht sind die beiden Werte für \(s_{opt}\), die sich bei Einhaltung des \(\beta\)-Servicegrades für die beiden Verteilungsfunktionen \(\Gamma(\alpha_{Y^*},k_{Y^*})\) und \(\mathcal{N}(0,1)\) ergeben.

Hierfür sind folgende von der Aufgabenstellung vorgegebenen Werte zu betrachten:
\begin{itemize}
	\item \(E(Y^*)=60\text{ Stück}\)
	\item \(Var(Y^*)\equiv\sigma(Y^*)=700\)
	\item \(\beta\text{-Servicegrad}=0,95\)
	\item \(q=100\text{ Stück}\) 
\end{itemize}

Der Wert für \(s_{opt}\) lässt sich durch eine Optimierungsstrategie bestimmen. Dabei entspricht \(s_{opt}\) dem niedrigsten Wert von \(s\) bei dem der vorgegebene Servicegrad \(\beta\) noch erreicht wird. 

Die dafür maximal zulässige  zu erwartende Fehlmenge bei einer zugrunde liegenden Normalverteilung, lässt sich über die zuvor genannte Formel für den \(\beta\)-Servicegrad berechnen:

\[\beta = 1-\frac{E\left(F_{\mathcal{N}(0,1)}(s)\right)}{q_{opt}} \Leftrightarrow E\left(F_{\mathcal{N}(0,1)}(s)\right)=(1-\beta)\cdot q_{opt}\]

Durch Einsetzen der vorgegebenen Werte für \(q\) und \(\beta\)-Servicegrad in die obige Gleichung, ergibt sich eine maximal zulässige Fehlmenge von \((1-0,95)\cdot 100 \text{ Stück}=5\text{ Stück}\). Dieser Wert gilt ebenfalls bei einer gammaverteilten Zufallsvariable.

Um den optimalen Bestellpunkt \(s_{opt}\) ermitteln zu können, müssen daher für verschiedene Werte von \(s\) die sich dadurch ergebenden Fehlmengen betrachtet werden und möglichst nahe an den zuvor berechneten Wert von 5 Stück angenähert werden.

Für die Normalverteilung kann der Erwartungswert für die Fehlmenge direkt durch die nachfolgende Gleichung berechnet werden. 

\[E\left(F_{\mathcal{N}(0,1)}(s)\right)=\Phi^1_{\mathcal{N}(0,1)}\left(\frac{s-E(Y^*)}{\sigma(Y^*)}\right)-\Phi^1_{\mathcal{N}(0,1)}\left(\frac{s+q_{opt}-E(Y^*)}{\sigma(Y^*)}\right)\]

Beim Einsetzen eines exemplarischen Wertes für s von 75 Stück, ergibt sich:

\begin{align*}
E\left(F_{\mathcal{N}(0,1)}(75)\right)&=\Phi^1_{\mathcal{N}(0,1)}\left(\frac{(75-60)\text{ Stück}}{700}\right)-\Phi^1_{\mathcal{N}(0,1)}\left(\frac{(75+100-60)\text{ Stück}}{700}\right)\\
&\approx4,7\text{ Stück}
\end{align*}

